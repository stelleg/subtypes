\documentclass[final]{beamer} 
\mode<presentation>
\usepackage{inconsolata}
\renewcommand*\familydefault{\ttdefault}
\usepackage{dejavu}
\renewcommand*\familydefault{\sfdefault}
\usepackage{lmodern}
\renewcommand\mathfamilydefault{\rmdefault}

\usepackage[T1]{fontenc}
\usepackage{listings}
\lstset{language=Haskell,basicstyle=\ttfamily,xleftmargin=1cm,aboveskip=0.5cm,belowskip=0.5cm}
\usepackage[size=a1]{beamerposter}
\usepackage{verbatim}
\usepackage{tikz}
\usetikzlibrary{arrows, fit, shapes}
\setbeamertemplate{headline}{  
  \leavevmode
  \begin{beamercolorbox}[wd=\paperwidth]{headline}
    \begin{columns}[T]
      \begin{column}{.05\paperwidth}
      \end{column}
      \begin{column}{.7\paperwidth}
        \vskip3cm
        \raggedright
        {\color{red}{{\LARGE{\inserttitle}}\\[1ex]}}
        \vskip0.5cm
        {\color{fg}{\large{\insertauthor}\\[1ex]}}
        %{\color{fg}{\large{\insertinstitute}\\[1ex]}}
      \end{column}
      \begin{column}{.25\paperwidth}
        \begin{center}
          \includegraphics[width=.7\linewidth]{unm}
        \end{center}
        \vskip2cm
      \end{column}
    \end{columns}
  \end{beamercolorbox}

  \begin{beamercolorbox}[wd=\paperwidth]{lower separation line head}
    \rule{0pt}{2pt}
  \end{beamercolorbox}
}

\setbeamertemplate{footline}{}

\mode<all>

\definecolor{red}{RGB}{205,16,65}
\definecolor{grey}{RGB}{109,111,113}

\setbeamerfont{structure}{family=\sffamily}
\setbeamercolor{structure}{fg=red}
\setbeamerfont{normal text}{family=\sffamily}
\setbeamercolor{normal text}{fg=grey,bg=white}

\title{Subtypes for Free!}
\author{George Stelle}
\institute{University of New Mexico}
%\date{\today} 

\begin{document}
\begin{frame}[fragile]
\begin{columns}
\begin{column}{.285\textwidth}

\begin{block}{Introduction}
\vspace{0.5cm}
Some runtime errors are caused by partial functions which are only defined for a
subtype of their input type, e.g. \texttt{head}. We present an approach to
implementing algebraic data types (ADTs) using Scott encodings types with GHC
that supports subtype inference to help prevent these kinds of failures.
\end{block}
\vspace{2cm}
\begin{block}{Subtypes from Parametric Polymorphism}
\vspace{0.5cm}
Parametric polymorphism implicitly defines subtype lattices. For example, given
the type $\forall a, a \rightarrow a \rightarrow a$, we can derive the following
subtype lattice:
\vspace{0.5cm}
\begin{figure}[!h]
\centering
\begin{tikzpicture}[->, line width=2pt, auto, node distance=4cm]
\node (1) at (5,10) {$\forall a, a \rightarrow a \rightarrow a$};
\node (2) at (0,5) {$\forall a b, a \rightarrow b \rightarrow a$};
\node (3) at (10,5) {$\forall a b, a \rightarrow b \rightarrow b$};
\node (4) at (5,0) {$\bot$};

\path (2) edge node {} (1)
      (3) edge node {} (1)
      (4) edge node {} (2)
      (4) edge node {} (3);
\end{tikzpicture}
\end{figure}
By subtype relation, we mean the following: a type $\tau$ is a subtype of of
another type $\tau'$ iff every value of type $\tau$ is also of type $\tau'$.
\end{block}
\vspace{2cm}
\begin{block}{Unification is Join}
\vspace{0.5cm}
Unification-based type inference has the useful property that it implements the
join (least upper bound) for the lattice described above. Intuitively, this
means that the types inferred will be as specific as possible, given the type
inference algorithm. 
\end{block}
\vspace{2cm}
\begin{block}{Scott Encodings}
\vspace{0.5cm}
Scott encodings are a way of implementing algebraic data types (ADTs) using
lambda terms. They enable the use of subtypes inherent in parametric
polymorphism to implement subtypes for ADTs. 
\end{block}
\end{column}

\begin{column}{.33\textwidth}
\begin{block}{Example: A Total \texttt{fromJust}}
\vspace{0.5cm}
The Haskell data type

\begin{verbatim}
  data Maybe a = Nothing
               | Just a
\end{verbatim}

can be implemented using Scott encodings as follows:

\begin{verbatim}
  newtype Maybe' a n j m = Maybe (n -> (a -> j) -> m)
  type Maybe a = forall m. Maybe' a m m m 
  type Just a = forall n j. Maybe' a n j j
  just :: a -> Just a
  just a = Maybe $ \n j -> j a
  type Nothing = forall a n j. Maybe' a n j n
  nothing :: Nothing
  nothing = Maybe $ \n j -> n
\end{verbatim}

We wrap the Scott encoding in a \texttt{newtype} to preserve type safety and
enable implementation of recursive types. This definition of \texttt{Maybe}
gives us the following subtype lattice:

\begin{figure}[!h]
\centering
\begin{tikzpicture}[->, line width=2pt, auto, node distance=4cm]
\node (1) at (5,10) {\texttt{Maybe a}};
\node (2) at (0,5) {\texttt{Nothing}};
\node (3) at (10,5) {\texttt{Just a}};
\node (4) at (5,0) {\texttt{Void}};

\path (2) edge node {} (1)
      (3) edge node {} (1)
      (4) edge node {} (2)
      (4) edge node {} (3);
\end{tikzpicture}
\end{figure}

The function \texttt{fromJust} in the Haskell standard library is a
\emph{partial} function: it can fail at runtime. We use the above types to
define a total version of \texttt{fromJust} that is type-safe, i.e. cannot
fail at runtime.

\begin{verbatim}
  bottom :: Void
  bottom = undefined

  fromJust :: Just a -> a
  fromJust (Maybe j) = j bottom id 
\end{verbatim}

We have prevented a class of runtime error, using the power of parametric
polymorphism!

\end{block}

\end{column}

\begin{column}{.285\textwidth}
\begin{block}{Other Examples}
\vspace{0.5cm}
There are number of other nice properties expressible and inferable with this
approach. For example:

\begin{verbatim}
   true `and` false :: False
   not . not $ true :: True
   [true, false] :: Boolean
   head :: Cons a -> a
   tail :: Cons a -> List a
   null nil :: True
   wizard :: DaggerOrStaff -> Wizard
\end{verbatim}

Another example application of this approach is the implementation of the
canonical example for GADTs and phantom types of a type-safe evaluator:

\begin{verbatim}
  type IntTerm = forall i ni. Term' i i ni ni i
  ...
  add :: IntTerm -> IntTerm -> IntTerm
  tnot :: BoolTerm -> BoolTerm 
\end{verbatim}

In a similar vein, we can define evaluators which return a value, where a value
is a subtype of the expression type, e.g.:

\begin{verbatim}
  eval :: Expr -> Value
\end{verbatim}

This is a common pattern, and important for the type safety of evaluators. 

\end{block}
\vspace{2cm}
\begin{block}{Current Drawbacks and Limitations}
\vspace{0.5cm}
\begin{itemize}
\item No recursive subtypes, e.g. can't write \verb!head . tail!
\item Potential performance impacts
\item Ugly: No pattern matching, verbose data type declarations 
\item Type composition: Needs impredicative types
\item Lacks correctness proofs
\end{itemize}
\vspace{2cm}
\end{block}

\begin{block}{Conclusion}
\vspace{0.5cm}
The restricted form of subtyping shown here could help Haskell programs avoid
some classes of run-time errors by defining total versions of existing partial
functions. 
\end{block}
\end{column}
\end{columns}
\end{frame}
\end{document}
