\documentclass[final]{beamer} 
\mode<presentation>
\usepackage{inconsolata}
\renewcommand*\familydefault{\ttdefault}
\usepackage{dejavu}
\renewcommand*\familydefault{\sfdefault}
\usepackage{lmodern}
\renewcommand\mathfamilydefault{\rmdefault}

\usepackage[T1]{fontenc}
\usepackage{listings}
\lstset{language=Haskell,basicstyle=\ttfamily,xleftmargin=1cm,aboveskip=0.5cm,belowskip=0.5cm}
\usepackage[size=a1]{beamerposter}
\usepackage{verbatim}
\usepackage{tikz}
\usetikzlibrary{arrows, fit, shapes}
\setbeamertemplate{headline}{  
%  \leavevmode
  \begin{beamercolorbox}[wd=\paperwidth]{headline}
    \begin{columns}[T]
      \begin{column}{.05\paperwidth}
      \end{column}
      \begin{column}{.2\paperwidth}
        \vskip1.5cm
        \raggedright
        {\color{lightgrey}{\large{\insertauthor}\\[1ex]}}
        {\color{lightgrey}{\normalsize{\insertinstitute}\\[1ex]}}
        {\color{lightgrey}{\normalsize{\texttt{stelleg@cs.unm.edu}}\\[1ex]}}
      \end{column}
      \begin{column}{.5\paperwidth}
        \vskip2cm
        \centering
        {\color{red}{{\Huge{\textbf{\inserttitle}}}\\[1ex]}}
      \end{column}
      \begin{column}{.25\paperwidth}
        \begin{center}
          \includegraphics[width=.7\linewidth]{unm}
        \end{center}
        %\vskip1cm
      \end{column}
    \end{columns}
  \end{beamercolorbox}
%  \begin{beamercolorbox}[wd=\paperwidth]{lower separation line head}
%    \rule{0pt}{2pt}
%  \end{beamercolorbox}
}

\setbeamertemplate{footline}{}

\mode<all>

\definecolor{red}{RGB}{205,16,65}
\definecolor{grey}{RGB}{25,25,25}
\definecolor{lightgrey}{RGB}{109,111,113}

\setbeamerfont{structure}{family=\sffamily}
\setbeamercolor{structure}{fg=red}
\setbeamerfont{normal text}{family=\sffamily}
\setbeamercolor{normal text}{fg=grey,bg=white}

\title{Subtypes for Free!}
\author{George Stelle}
\institute{Department of Computer Science \\ University of New Mexico}
%\date{\today} 

\begin{document}
\begin{frame}[fragile]
\begin{columns}[t]
\begin{column}{.285\textwidth}

\begin{block}{Introduction}
\vspace{0.5cm}
Some runtime errors are caused by partial functions which are only defined for a
subtype of their input type, e.g. \texttt{head}. We present a simple subtyping
scheme enabled by parametric polymorphism that could help Haskell programs avoid
some of these run-time errors by defining total versions of existing partial
functions. 
\end{block}
\vspace{1cm}
\begin{block}{Subtypes from Parametric Polymorphism}
\vspace{0.5cm}
\textbf{Claim}: Parametric polymorphism implicitly defines subtype lattices. For example, given
the type $\forall a, a \rightarrow a \rightarrow a$, we can derive the following
subtype lattice:
\vspace{0.5cm}
\begin{figure}[!h]
\centering
\begin{tikzpicture}[->, line width=2pt, auto, node distance=4cm]
\node (1) at (5,10) {$\forall a, a \rightarrow a \rightarrow a$};
\node (2) at (0,5) {$\forall a b, a \rightarrow b \rightarrow a$};
\node (3) at (10,5) {$\forall a b, a \rightarrow b \rightarrow b$};
\node (4) at (5,0) {$\forall a b c, a \rightarrow b \rightarrow c$};

\path (2) edge node {} (1)
      (3) edge node {} (1)
      (4) edge node {} (2)
      (4) edge node {} (3);
\end{tikzpicture}
\end{figure}
Subtype relation (subsumption): a type $\tau$ is a subtype of another type $\tau'$ iff every
expression of type $\tau$ is also of type $\tau'$.
\end{block}
\vspace{1cm}
\begin{block}{Unification is Join}
\vspace{0.5cm}
Unification-based type inference implements join (least upper bound) for the
lattice described above. Intuitively, types inferred will be as specific as
possible. 
\end{block}
\vspace{1cm}
\begin{block}{Scott Encodings}
\vspace{0.5cm}
Scott encodings are a method for implementing algebraic data types (ADTs)
using lambda terms. For any algebraic data type with $m$ constructors,
constructor $C_i$ with $n$ parameters is implemented as:

$$C_i \tau_0 \tau_1 \cdots \tau_n = \lambda x_0 x_1 \cdots x_n c_0 c_1 \cdots c_m . c_i x_0 x_1 \cdots x_n$$
\end{block}
\end{column}

\begin{column}{.33\textwidth}
\begin{block}{Example: \texttt{fromJust}}
\vspace{0.5cm}
We begin with the Haskell data type for \texttt{Maybe}:

\begin{verbatim}
  data Maybe a = Nothing
               | Just a
\end{verbatim}

A useful function defined in the Haskell standard library for this type is the
function \verb!fromJust :: Maybe a -> a!, which fails when applied to
\texttt{Nothing}.

\begin{verbatim}
  fromJust Nothing = error "Maybe.fromJust: Nothing"
\end{verbatim}

We show how to define a total version of \texttt{fromJust}, with type
\texttt{Just a -> a}. To start, we replace \texttt{Maybe} using Scott
encodings:

\begin{verbatim}
  type Maybe a = forall m. m -> (a -> m) -> m 
  type Just a = forall n j. n -> (a -> j) -> j
  just :: a -> Just a
  just a = \n j -> j a
  type Nothing = forall a n j. n -> (a -> j) -> n
  nothing :: Nothing
  nothing = \n j -> n
  type MaybeBottom = forall a n j m. n -> (a -> j) -> m
\end{verbatim}

This definition of \texttt{Maybe} gives us the following subtype lattice:

\begin{figure}[!h]
\centering
\begin{tikzpicture}[->, align=center, line width=2pt, auto, node distance=4cm]
\node (1) at (5,10) {\texttt{Maybe $a$} = \\ $\forall m, m \rightarrow (a
\rightarrow m) \rightarrow m$ \\ \{\texttt{just a}, \texttt{nothing}\}};
\node (2) at (-2,5) {\texttt{Just $a$} = \\ $\forall n j, n \rightarrow (a \rightarrow
j) \rightarrow j$ \\ \{\texttt{just a}\} };
\node (3) at (12,5) {\texttt{Nothing} = \\ $\forall a n j, n \rightarrow (a
\rightarrow j) \rightarrow n$ \\ \{\texttt{nothing}\} };
\node (4) at (5,0) {\texttt{MaybeBottom} = \\ $\forall a n j m, n \rightarrow (a \rightarrow j)
\rightarrow m$ \\ \{\} };

\path (2) edge node {} (1)
      (3) edge node {} (1)
      (4) edge node {} (2)
      (4) edge node {} (3);
\end{tikzpicture}
\end{figure}

The function \texttt{fromJust} in the Haskell standard library is a
\emph{partial} function: it can fail at runtime. We replace it with a total
version that is type-safe, i.e. cannot fail at runtime:

\begin{verbatim}
  fromJust :: Just a -> a
  fromJust j = j (undefined::Void) id 
\end{verbatim}

\end{block}

\end{column}

\begin{column}{.285\textwidth}
\begin{block}{Other Examples}
\vspace{0.5cm}

\begin{verbatim}
 true `and` false :: False
 not . not $ true :: True
 [true, true] :: [True]
 [true, false] :: [Bool]
 head :: Cons a -> a
 tail :: Cons a -> List a
 null nil :: True
 wizard :: DaggerOrStaff -> Wizard
 add :: IntTerm -> IntTerm -> IntTerm
 tnot :: BoolTerm -> BoolTerm 
 eval :: Expr -> Value
\end{verbatim}

\textbf{Implementation:} These and other examples implemented at
\texttt{http://cs.unm.edu/\textasciitilde stelleg/subtypes}

\end{block}
\vspace{0.5cm}
\begin{block}{Related Work}
\vspace{0.5cm}
Informally in descending order in terms of expressive power:

\begin{itemize}
\item Dependent Types
\item Refinement Types
\item GADTs / Phantom Types
\item OO Subtypes / Case Classes
\item Polymorphic Variants
\item {\color{red} Free Subtypes}
\item Hindley Milner
\end{itemize}
\begin{comment}
\begin{center}
\begin{tikzpicture}[-, line width=1pt, auto, node distance=4cm]

\node (5) at (0,-4) {Hindley Milner};
\node (0) at (0,-2) {\color{red} Free Subtypes};
\node (6) at (0,0) {Polymorphic Variants};
\node (1) at (0,2) {OO Subtypes / Case Classes};
\node (2) at (0,4) {GADTs / Phantom Types};
\node (3) at (0,6) {Refinement Types};
\node (4) at (0,8) {Dependent Types};

\path (0) edge node {} (6)
      (6) edge node {} (1)
      (1) edge node {} (2)
      (2) edge node {} (3)
      (3) edge node {} (4)
      (5) edge node {} (0);
\end{tikzpicture}
\end{center}
\end{comment}

\textbf{Big Picture}: Each approach represents a \emph{compromise}.

\end{block}
\vspace{1.5cm}
\begin{block}{Limitations and Future Work}
\vspace{0.5cm}
\begin{itemize}
\item Type composition: Needs impredicative types \textbf{*}
\item Type classes: Too polymorphic \textbf{*}
\item Potential performance impacts: Compile and run time \textbf{*}
\item No pattern matching, verbose data type declarations \textbf{*}
\item Readable types $\oplus$ Composable inference : \verb!  not :: Bool -> Bool!
\item Recursive subtypes: Can't write \verb!head . tail!
\end{itemize}
\textbf{* Compiler support can help.}
\end{block}

\end{column}
\end{columns}
\end{frame}
\end{document}
