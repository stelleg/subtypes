\documentclass[]{article}
\usepackage[a4paper, margin=1in]{geometry}
\usepackage{lmodern}
\usepackage{amssymb,amsmath}
\usepackage{ifxetex,ifluatex}
\usepackage{fixltx2e} % provides \textsubscript
\ifnum 0\ifxetex 1\fi\ifluatex 1\fi=0 % if pdftex
  \usepackage[T1]{fontenc}
  \usepackage[utf8]{inputenc}
\else % if luatex or xelatex
  \ifxetex
    \usepackage{mathspec}
    \usepackage{xltxtra,xunicode}
  \else
    \usepackage{fontspec}
  \fi
  \defaultfontfeatures{Mapping=tex-text,Scale=MatchLowercase}
  \newcommand{\euro}{€}
\fi
% use upquote if available, for straight quotes in verbatim environments
\IfFileExists{upquote.sty}{\usepackage{upquote}}{}
% use microtype if available
\IfFileExists{microtype.sty}{%
\usepackage{microtype}
\UseMicrotypeSet[protrusion]{basicmath} % disable protrusion for tt fonts
}{}
\ifxetex
  \usepackage[setpagesize=false, % page size defined by xetex
              unicode=false, % unicode breaks when used with xetex
              xetex]{hyperref}
\else
  \usepackage[unicode=true]{hyperref}
\fi
\hypersetup{breaklinks=true,
            bookmarks=true,
            pdfauthor={George Stelle and Stephanie Forrest},
            pdftitle={Subtypes for Free!},
            colorlinks=true,
            citecolor=blue,
            urlcolor=blue,
            linkcolor=magenta,
            pdfborder={0 0 0}}
\urlstyle{same}  % don't use monospace font for urls
\usepackage{color}
\usepackage{fancyvrb}
\newcommand{\VerbBar}{|}
\newcommand{\VERB}{\Verb[commandchars=\\\{\}]}
\DefineVerbatimEnvironment{Highlighting}{Verbatim}{commandchars=\\\{\}}
% Add ',fontsize=\small' for more characters per line
\newenvironment{Shaded}{}{}
\newcommand{\KeywordTok}[1]{\textcolor[rgb]{0.00,0.44,0.13}{\textbf{{#1}}}}
\newcommand{\DataTypeTok}[1]{\textcolor[rgb]{0.56,0.13,0.00}{{#1}}}
\newcommand{\DecValTok}[1]{\textcolor[rgb]{0.25,0.63,0.44}{{#1}}}
\newcommand{\BaseNTok}[1]{\textcolor[rgb]{0.25,0.63,0.44}{{#1}}}
\newcommand{\FloatTok}[1]{\textcolor[rgb]{0.25,0.63,0.44}{{#1}}}
\newcommand{\CharTok}[1]{\textcolor[rgb]{0.25,0.44,0.63}{{#1}}}
\newcommand{\StringTok}[1]{\textcolor[rgb]{0.25,0.44,0.63}{{#1}}}
\newcommand{\CommentTok}[1]{\textcolor[rgb]{0.38,0.63,0.69}{\textit{{#1}}}}
\newcommand{\OtherTok}[1]{\textcolor[rgb]{0.00,0.44,0.13}{{#1}}}
\newcommand{\AlertTok}[1]{\textcolor[rgb]{1.00,0.00,0.00}{\textbf{{#1}}}}
\newcommand{\FunctionTok}[1]{\textcolor[rgb]{0.02,0.16,0.49}{{#1}}}
\newcommand{\RegionMarkerTok}[1]{{#1}}
\newcommand{\ErrorTok}[1]{\textcolor[rgb]{1.00,0.00,0.00}{\textbf{{#1}}}}
\newcommand{\NormalTok}[1]{{#1}}
\usepackage[normalem]{ulem}
% avoid problems with \sout in headers with hyperref:
\pdfstringdefDisableCommands{\renewcommand{\sout}{}}
\setlength{\parindent}{0pt}
\setlength{\parskip}{6pt plus 2pt minus 1pt}
\setlength{\emergencystretch}{3em}  % prevent overfull lines
\providecommand{\tightlist}{%
  \setlength{\itemsep}{0pt}\setlength{\parskip}{0pt}}
\setcounter{secnumdepth}{0}

\title{Subtypes for Free!}
\author{George Stelle and Stephanie Forrest}
\date{}

% Redefines (sub)paragraphs to behave more like sections
\ifx\paragraph\undefined\else
\let\oldparagraph\paragraph
\renewcommand{\paragraph}[1]{\oldparagraph{#1}\mbox{}}
\fi
\ifx\subparagraph\undefined\else
\let\oldsubparagraph\subparagraph
\renewcommand{\subparagraph}[1]{\oldsubparagraph{#1}\mbox{}}
\fi

\begin{document}
\maketitle

\section{Motivation}\label{motivation}

It would sometimes be nice if a type system could be as precise as
possible to restrict what a value will be. For example, the type
\texttt{Bool} ensures a value will either be \texttt{true} or
\texttt{false}, but it doesn't know which. What we want is the type
system to be precise when possible, so instead of always saying
\texttt{Bool} (or ``I don't know''), it could say \texttt{True},
\texttt{False}, or \texttt{Bool}. In this example, \texttt{True} and
\texttt{False} are \emph{subtypes} of \texttt{Bool}, i.e.~every value of
type \texttt{True} is also of type \texttt{Bool}.

\section{Existing Approaches}\label{existing-approaches}

There is a significant literature on subtyping, mostly in the context of
object-oriented (OO) languages. Indeed, subtyping in languages that
combine OO features with functional features implement subtypes using
objects \cite{scala, ocaml}. We are interested in subtyping without
requiring all the language complexities introduced by OO constructs, and
will see that doing so gives us some inference advantages.

Generalized algebraic data types (GADTs) allow for a restricted form of
subtyping as well. For example, the canonical example of
\texttt{Expr\ Int} as separated from \texttt{Expr\ Bool} in a simple
language with booleans and integers. Our approach subsumes GADTs, while
adding extra flexibility in how we can represent subtypes.

\section{Our Approach}\label{our-approach}

Our approach is to use Scott encodings of algebraic data types, along
with Hindley Milner type inference, to achieve a very general form of
subtype polymorphism. We show how GHC can take advantage of this
approach without modification by \sout{abusing} taking advantage of
impredicative types.

Scott encodings are encodings of algebraic data types that encode their
own \texttt{case} destruction. For booleans, for example, the encoding
is identical to Church encoding: \texttt{true\ t\ f\ =\ t}, and
\texttt{false\ t\ f\ =\ f}.

Our primary insight is that by wrapping the scott encodings in a
\texttt{newtype}, and then carefully constraining the types, we can
define types that represent arbitrary subsets of the constructors.

One appealing thing about our approach is that it composes well. For
example, we are not aware of any other subtyping scheme that is capable
of inferring that

\begin{Shaded}
\begin{Highlighting}[]
   \NormalTok{null}\OtherTok{ nil ::} \DataTypeTok{True}
\end{Highlighting}
\end{Shaded}

where \texttt{null} checks if a list is empty and \texttt{nil} is the
empty list.

\section{An Example}\label{an-example}

One application of subtyping is to prevent partial functions. Here we
show how to use our approach to define a total version of the
\texttt{fromJust} from the Haskell standard library.

To achieve this, we define a subtypeable \texttt{Maybe}. The Scott
encoding for \texttt{Maybe} takes two \emph{case} arguments, the first,
\texttt{n}, corresponding to \texttt{nothing}, so it has no parameters,
and the second, \texttt{j}, corresponding to \texttt{just\ a}, so it has
one parameter of type \texttt{a}.

\begin{Shaded}
\begin{Highlighting}[]
  \KeywordTok{newtype} \DataTypeTok{Maybe'} \NormalTok{a n j m }\FunctionTok{=} \DataTypeTok{Maybe} \NormalTok{(n }\OtherTok{->} \NormalTok{(a }\OtherTok{->} \NormalTok{j) }\OtherTok{->} \NormalTok{m)}
  \KeywordTok{type} \DataTypeTok{Maybe} \NormalTok{a }\FunctionTok{=} \NormalTok{forall m}\FunctionTok{.} \DataTypeTok{Maybe'} \NormalTok{a m m m }
  \KeywordTok{type} \DataTypeTok{Just} \NormalTok{a }\FunctionTok{=} \NormalTok{forall n j}\FunctionTok{.} \DataTypeTok{Maybe'} \NormalTok{a n j j}
\OtherTok{  just ::} \NormalTok{a }\OtherTok{->} \DataTypeTok{Just} \NormalTok{a}
  \NormalTok{just a }\FunctionTok{=} \DataTypeTok{Maybe} \FunctionTok{$} \NormalTok{\textbackslash{}n j }\OtherTok{->} \NormalTok{j a}
  \KeywordTok{type} \DataTypeTok{Nothing} \FunctionTok{=} \NormalTok{forall a n j}\FunctionTok{.} \DataTypeTok{Maybe'} \NormalTok{a n j n}
\OtherTok{  nothing ::} \DataTypeTok{Nothing}
  \NormalTok{nothing }\FunctionTok{=} \DataTypeTok{Maybe} \FunctionTok{$} \NormalTok{\textbackslash{}n j }\OtherTok{->} \NormalTok{n}
\end{Highlighting}
\end{Shaded}

For our \texttt{Maybe} type, we have a type like the standard
\texttt{Data.Maybe} type in Haskell. In this case, similar to the
\texttt{maybe} function from \texttt{Data.Maybe}, we don't know whether
we have a value of type \texttt{Just} or a value of type
\texttt{Nothing}, so we must ensure that the two cases return the same
type. But if the compiler can infer that the value is of type
\texttt{Just}, we don't care what the \texttt{n} cases type is, and only
constrain the return type to be the same as the \texttt{Just} case
\texttt{j}.

With that in mind, we can write our total function, using a
\texttt{Bottom} type with no values to convince ourselves that our
function will never typecheck \texttt{fromJust} applied to a type that
includes \texttt{Nothing}.

\begin{Shaded}
\begin{Highlighting}[]
\OtherTok{  fromJust ::} \DataTypeTok{Just} \NormalTok{a }\OtherTok{->} \NormalTok{a}
  \NormalTok{fromJust (}\DataTypeTok{Maybe} \NormalTok{j) }\FunctionTok{=} \NormalTok{j bottom }\FunctionTok{$} \NormalTok{\textbackslash{}a}\OtherTok{->}\NormalTok{a}

  \KeywordTok{data} \DataTypeTok{Bottom}
  \NormalTok{bottom }\FunctionTok{=} \NormalTok{error }\StringTok{"impossible"}\OtherTok{ ::} \DataTypeTok{Bottom}
\end{Highlighting}
\end{Shaded}

We can still define the partial version of \texttt{fromJust}, which like
the one from the Haskell standard library, allows runtime failure.

\begin{Shaded}
\begin{Highlighting}[]
\OtherTok{  partialFromJust ::} \DataTypeTok{Maybe} \NormalTok{a }\OtherTok{->} \NormalTok{a}
  \NormalTok{partialFromJust (}\DataTypeTok{Maybe} \NormalTok{m) }\FunctionTok{=} \NormalTok{m (error }\StringTok{"partialFromJust Nothing"}\NormalTok{) }\FunctionTok{$} \NormalTok{\textbackslash{}a}\OtherTok{->}\NormalTok{a }
\end{Highlighting}
\end{Shaded}

We have implemented more sophisticated examples, including recursive
types, GADTs, and total versions of \texttt{head} and \texttt{tail}. For
these, as well as an implementation of
\href{http://ericlippert.com/2015/04/27/wizards-and-warriors-part-one/}{Eric
Lippert's wizards and warriors example}, see:
\href{http://cs.unm.edu/~stelleg/Scott/}{http://cs.unm.edu/\textasciitilde{}stelleg/scott/}.

\section{Drawbacks}\label{drawbacks}

Without language support, this approach is quite verbose and unwieldy.
Furthermore, due to the nature of Scott encodings, the constructor
definitions grow quadratically in size as a function of the number of
constructors. Because of the large number of nested \texttt{forall}s,
getting variable quantification right can also be quite difficult.

\section{Future Work}\label{future-work}

We are currently working on formalizing and verifying this approach in
Coq using parametricity \cite{parametricity}. The burden of creating all
of the necessary type synonyms could likely be lessened by some template
Haskell. Further in the future, it would be interesting to see the ideas
integrated as an extension to GHC.

\end{document}
